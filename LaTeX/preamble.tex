% !TEX root = main.tex
%
% TODO: Have a clearer separation between ZHAW Thesis class and this preamble.

%%%%%%%%%%%%%%%%%%%%%%%%%%%%%%%%%%%%%%%%%%%%%%%%%%%%%%%%%%%%%%%%%%%%%%%%%%%%%%%%
% Fonts and characters
%%%%%%%%%%%%%%%%%%%%%%%%%%%%%%%%%%%%%%%%%%%%%%%%%%%%%%%%%%%%%%%%%%%%%%%%%%%%%%%%

% Support for special characters
\usepackage[utf8]{inputenc}    % Input encoding: Support non-ascii characters
\usepackage[T1]{fontenc}       % Font encoding: T1 is good choice for Europe

% Set main fonts
% Fonts catalogue: https://tug.org/FontCatalogue/
\usepackage{mathpazo}          % Use the Palatino font by default
\usepackage{beramono}          % Override the monospace/typewriter font

\usepackage{amssymb}           % Maths
\usepackage{amsmath}           % Maths 
\usepackage{numprint}          % For nicely printing large numbers
\usepackage{lmodern}           % Use LaTeX math fonts, requires T1
\usepackage[normalem]{ulem}    % Strikethrough text.

\usepackage{pdfpages}

% ZHAW title font
% Try to load Helvetica Rounded Bold, and OpenType font.
% Loading OTF or system fonts is possible with XeLaTeX.
% If the document is compiled using pdfLaTeX, resort 
\usepackage{ifxetex}
\ifxetex
    \usepackage{fontspec}
    \newfontfamily\zhawtitlefont{Helvetica Rounded Bold}
\else
    \newcommand{\zhawtitlefont}{\scshape}
\fi

\newcommand{\boldit}[1]{\textit{\textbf{#1}}}

% Set Helvetica as the default font
\usepackage[scaled]{helvet}
\renewcommand{\familydefault}{\sfdefault}

%%%%%%%%%%%%%%%%%%%%%%%%%%%%%%%%%%%%%%%%%%%%%%%%%%%%%%%%%%%%%%%%%%%%%%%%%%%%%%%%
% Comments
%%%%%%%%%%%%%%%%%%%%%%%%%%%%%%%%%%%%%%%%%%%%%%%%%%%%%%%%%%%%%%%%%%%%%%%%%%%%%%%%

\usepackage[color=cyan,colorinlistoftodos]{todonotes}
\newcommand{\replace}[2]{{\sout{#1}} \textcolor{cyan}{#2}}
\newcommand{\note}[1]{\textcolor{cyan}{#1}}

%%%%%%%%%%%%%%%%%%%%%%%%%%%%%%%%%%%%%%%%%%%%%%%%%%%%%%%%%%%%%%%%%%%%%%%%%%%%%%%%
% Colors
%%%%%%%%%%%%%%%%%%%%%%%%%%%%%%%%%%%%%%%%%%%%%%%%%%%%%%%%%%%%%%%%%%%%%%%%%%%%%%%%

% Set up colors
\usepackage[dvipsnames]{xcolor} % Color management
% \PassOptionsToPackage{dvipsnames}{xcolor}  % Had to change this.

% ZHAW Blue: Pantone 2945 U / R0 G100 B166
\definecolor{zhawblue}{rgb}{0.00, 0.39, 0.65}
\definecolor{zhawlightblue}{rgb}{0.82, 0.88, 0.93}
% Colors related to code listings
\definecolor{codegreen}{rgb}{0,0.6,0}
\definecolor{codegray}{rgb}{0.5,0.5,0.5}
\definecolor{codepurple}{rgb}{0.58,0,0.82}
\definecolor{codebackground}{rgb}{0.93,0.94,0.95}
% Colors related to tables
\definecolor{tablegray}{rgb}{0.9,0.9,0.9}
\colorlet{tableheader}{tablegray}
% Colors related to hyperref
\definecolor{linkcolor}{rgb}{0.00, 0.39, 0.65}
\definecolor{urlcolor}{rgb}{0.00, 0.39, 0.65}

%%%%%%%%%%%%%%%%%%%%%%%%%%%%%%%%%%%%%%%%%%%%%%%%%%%%%%%%%%%%%%%%%%%%%%%%%%%%%%%%
% Environments
%%%%%%%%%%%%%%%%%%%%%%%%%%%%%%%%%%%%%%%%%%%%%%%%%%%%%%%%%%%%%%%%%%%%%%%%%%%%%%%%

% \usepackage[ngerman]{babel}   % Language support
\usepackage[english]{babel}     % Language support
\usepackage[                    % Create hypertext links
    colorlinks=true, 
    allbordercolors={white}, 
    linkcolor=linkcolor,
    citecolor=linkcolor, 
    filecolor=linkcolor,
    linktoc=all, 
    urlcolor=urlcolor
    ]{hyperref}

\usepackage{xurl}              % Superior to url or breakurl for breaking long URLs
\usepackage{graphicx}          % Include graphics
\usepackage{float}             % Improved interface for floating objects
\usepackage{tabularx}          % Create nicer tables
\usepackage{multirow}          % Merge rows in tables
\usepackage{booktabs}          % To further beautify tables
\usepackage{longtable}         % For tables that span multiple pages
\usepackage{caption}           % Customized caption
\usepackage{subcaption}        % Subfigure captions
\usepackage{makecell}          % Per-cell formatting in tables (\makecell)
\usepackage{pdfpages}          % Required to include PDF files/graphics (\includepdf)
\usepackage{colortbl}          % Tables with colors
\usepackage[tooltip]{acro}     % For acronyms and abbreviations
\usepackage[autostyle=true]{csquotes} 
                               % Required to generate language-dependent quotes 
                               % in the bibliography
  
\usepackage{todonotes}         % Introduces the command \todo
\setlength{\marginparwidth}{2.5cm} % Adjust this if the todo notes are out of margins
\usepackage{array,ragged2e}    % For ragged alignment of multi-line table cells

% Adjust tables
\colorlet{tableheadcolor}{gray!25}                % Table header colour
\newcommand{\headcol}{\rowcolor{tableheadcolor}}% % The color of the head
\newcommand{\topline}{\arrayrulecolor{black}      %
                      \specialrule{0.1em}{\abovetopsep}{0.5pt}%
                      \arrayrulecolor{tableheadcolor}%
                      \specialrule{\belowrulesep}{0pt}{-3pt}%
            \arrayrulecolor{black}
            }
\newcommand{\midline}{\arrayrulecolor{tableheadcolor}
            \specialrule{\aboverulesep}{-1pt}{0pt}%
            \arrayrulecolor{black}\specialrule{\lightrulewidth}{0pt}{0pt}%
            \arrayrulecolor{white}\specialrule{\belowrulesep}{0pt}{-3pt}%
            \arrayrulecolor{black}
            }
\newcommand{\bottomline}{\arrayrulecolor{white}\specialrule{\aboverulesep}{0pt}{-2pt}%
            \arrayrulecolor{black}\specialrule{\heavyrulewidth}{0pt}{\belowbottomsep}}%

% Create boxes as follows:
% \begin{colorbox}{red}{2}
\usepackage{tcolorbox}
\newtcolorbox{textbox}[2]{
    arc=3pt,
    boxrule=#2pt,
    colback=#1!25!white,
    width=\textwidth,
    halign=left,
    valign=center,
    colframe=#1!75!black
}

%%%%%%%%%%%%%%%%%%%%%%%%%%%%%%%%%%%%%%%%%%%%%%%%%%%%%%%%%%%%%%%%%%%%%%%%%%%%%%%%
% Code listings
%%%%%%%%%%%%%%%%%%%%%%%%%%%%%%%%%%%%%%%%%%%%%%%%%%%%%%%%%%%%%%%%%%%%%%%%%%%%%%%%
\captionsetup{format=plain,             % plain or hang
              justification=justified,  % justified, centering, raggedright, ...
              labelfont=bf,
              font=small,
              margin=20pt}
              

%%%%%%%%%%%%%%%%%%%%%%%%%%%%%%%%%%%%%%%%%%%%%%%%%%%%%%%%%%%%%%%%%%%%%%%%%%%%%%%%
% Code listings
%%%%%%%%%%%%%%%%%%%%%%%%%%%%%%%%%%%%%%%%%%%%%%%%%%%%%%%%%%%%%%%%%%%%%%%%%%%%%%%%

\newcommand{\code}[1]{\texttt{#1}}



% Setup code listings
\usepackage{listings}
\lstdefinestyle{mystyle}{
    backgroundcolor=\color{codebackground},   
    commentstyle=\color{codegreen},
    keywordstyle=\color{magenta},
    numberstyle=\tiny\color{codegray},
    stringstyle=\color{codepurple},
    basicstyle=\ttfamily\scriptsize,
    breakatwhitespace=false,
    breaklines=true,
%    captionpos=b,
    keepspaces=true,
    numbers=left,
    numbersep=5pt,
    showspaces=false,
    showstringspaces=false,
    showtabs=false,
    tabsize=4
}
\lstset{style=mystyle}

% minted is an alternative code listing package. (See appendix A)
% For it to run successfully, ensure the following:
% - the Python package Pygments. Install with the following command:
%       python -m pip install Pygments
% - pdflatex (or xelatex) is executed with the flag --shell-escape
%   If you are using a TEX editor, you can modify the typesetting 
%   command somewhere in the settings.
%\usepackage[outputdir=build]{minted}
%\usemintedstyle{xcode}
% For fancier coloring schemes, see here:
% https://tex.stackexchange.com/questions/585582
% One could also create an own style in Pygments
% https://pygments.org/docs/styles/#creating-own-styles

%%%%%%%%%%%%%%%%%%%%%%%%%%%%%%%%%%%%%%%%%%%%%%%%%%%%%%%%%%%%%%%%%%%%%%%%%%%%%%%%
% MARGIN SETTINGS
%%%%%%%%%%%%%%%%%%%%%%%%%%%%%%%%%%%%%%%%%%%%%%%%%%%%%%%%%%%%%%%%%%%%%%%%%%%%%%%%
\usepackage{geometry}
\geometry{
    paper=a4paper,       % Change to letterpaper for US letter
    %inner=2.5cm,        % Inner margin
    %outer=3.8cm,        % Outer margin
    %top=1.5cm,          % Top margin
    bottom=4.cm,         % Bottom margin
    bindingoffset=.5cm,  % Binding offset
    %showframe,          % Show the type block of the page
}

% Other layout settings
\setlength\parindent{0em} % No indent
\setlength{\parskip}{1em}
%\usepackage{parskip}


\setlength{\intextsep}{30pt}   % Distance between image and text 
\usepackage{enumitem}          % Layout control for list environments (e.g, itemize)
\setlist{noitemsep}            % Suppress extra spaces between items
%\setlist{nosep}               % Suppress spaces before/after list environments


%----------------------------------------------------------------------------------------
%   PENALTIES
%----------------------------------------------------------------------------------------

\doublehyphendemerits=10000 % No consecutive line hyphens
\brokenpenalty=10000 % No broken words across columns/pages
\widowpenalty=9999 % Almost no widows at bottom of page
\clubpenalty=9999 % Almost no orphans at top of page
\interfootnotelinepenalty=9999 % Almost never break footnotes


%%%%%%%%%%%%%%%%%%%%%%%%%%%%%%%%%%%%%%%%%%%%%%%%%%%%%%%%%%%%%%%%%%%%%%%%%%%%%%%%
% MANAGE AUTHOR LISTS
%%%%%%%%%%%%%%%%%%%%%%%%%%%%%%%%%%%%%%%%%%%%%%%%%%%%%%%%%%%%%%%%%%%%%%%%%%%%%%%%

\usepackage[affilmode=count]{authors}


% Trick: Convert the setter into a getter.
\NewDocumentCommand{\pdfautorstring}{m}{\renewcommand{\pdfautorstring}{#1}}
\NewDocumentCommand{\projecttitle}{m}{\renewcommand{\projecttitle}{#1}}
\NewDocumentCommand{\projecttype}{m}{\renewcommand{\projecttype}{#1}}
\NewDocumentCommand{\projectcode}{m}{\renewcommand{\projectcode}{#1}}
\NewDocumentCommand{\projectdate}{m}{\renewcommand{\projectdate}{#1}}
\NewDocumentCommand{\keywords}{m}{\renewcommand{\keywords}{#1}}

\NewDocumentCommand{\university}{m}{\renewcommand{\university}{#1}}
\NewDocumentCommand{\department}{m}{\renewcommand{\department}{#1}}
\NewDocumentCommand{\institute}{m}{\renewcommand{\institute}{#1}}
\NewDocumentCommand{\group}{m}{\renewcommand{\group}{#1}}


%----------------------------------------------------------------------------------------
%   HEADERS AND FOOTERS
%----------------------------------------------------------------------------------------

% Notes:
% - In twoside mode, left and right side pages may differ
% - Pages: left = even, right = odd
%   Chapters usually are forced to start at odd pages.
% - \markboth{left}{right} how left and right pages are marked
% - \automark is a convenience command from the scrlayer-scrpage package. It issues 
%   that \markboth is called for every \chapter and \section command.
% - Historical note: scrlayer-scrpage is the successor of scrpage2
% - Refs:
%    \markboth:  https://tex.stackexchange.com/questions/198676
%    \automark:  https://tex.stackexchange.com/questions/34269/
%    \automark*: https://tex.stackexchange.com/questions/348550/
%
\usepackage[markcase=used]{scrlayer-scrpage}
\ifoot{}% Empty inner footer by default
\ofoot{}% Empty outer footer by default
\providepairofpagestyles{basicStyle}{
    % Header specs for style basicStyle
    \clearpairofpagestyles%
    % \automark[section]{section}
    % \ihead{\headmark}% Inner header
    % \ohead[\pagemark]{\pagemark}% Outer header
    \cfoot[\pagemark]{\pagemark}% Footer with pagemark in the middle
}
\pagestyle{basicStyle}

\providepairofpagestyles[basicStyle]{reportStyle}{%
    \automark*[subsection]{}% Override right marks with section title once one is set
                         % Otherwise, use the chapter title -> basicStyle
}
% \pagestyle{reportStyle}

\KOMAoption{headsepline}{false} % set true to get a line below the header
