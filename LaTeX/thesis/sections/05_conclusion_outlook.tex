% Indicate the main file. Must go at the beginning of the file.
% !TEX root = ../main.tex

%%%%%%%%%%%%%%%%%%%%%%%%%%%%%%%%%%%%%%%%%%%%%%%%%%%%%%%%%%%%%%%%%%%%%%%%%%%%%%%%
% 05_conclusion_outlook
%%%%%%%%%%%%%%%%%%%%%%%%%%%%%%%%%%%%%%%%%%%%%%%%%%%%%%%%%%%%%%%%%%%%%%%%%%%%%%%%


\section{Conclusion and Outlook}
\label{conclusion_outlook}

\subsection{Conclusion}

This thesis demonstrated the effectiveness of deep learning models for classifying small mammals from camera trap images.
The pretrained EfficientNet-B0 model provided superior classification accuracy, quickly converging and demonstrating robustness across validation folds.
The integration of automated detections via MegaDetector, while beneficial, revealed critical limitations, particularly concerning misclassifications and missed detections for visually similar or partially obscured species.
Despite these limitations, the applied approach substantially reduces manual effort in ecological monitoring, making it a promising tool for biodiversity conservation.


\subsection{Outlook}

Future enhancements should focus on addressing current limitations by introducing an explicit category for non-target species to improve classification accuracy and reduce erroneous predictions.
Additional data collection efforts are necessary to enable reliable detection and classification of rare species such as \textit{glis\_glis}.
Implementing data augmentation could further enhance model robustness and generalization capabilities.

Moreover, future camera trap deployments could benefit from standardized sequence lengths derived from typical camera settings, leveraging EXIF metadata to streamline processing and improve model consistency.
An application interface or integrated software solution could be developed to enhance usability and accessibility of the model outputs for conservation practitioners.
Finally, incorporating methods such as Integrated Gradients for decision analysis could provide valuable insights into model predictions, supporting transparency and interpretability.

- implement other category - is it possible with no data for other or do i need data in order to implement it?

- generating some more data to make glis\_glis detectable

- creating a application for the model to make it usable

- Integrated Gradients - decision analysis

- The fact that the background of the images taken in the MammaliaBox is relatively uniform is not yet fully exploited.



new data generated info to discuss:
It is worth noting that in future use cases, the sequence length will likely be more standardized.
The actual length will depend on the camera settings -- common settings such as 1, 3, 5, or 10 images per trigger -- which can be extracted from the EXIF information of the images.

Data augmentation a well established way to improve model's generalization \autocite{shortenSurveyImageData2019} was implemented and considered as an option but not actually used in the end.