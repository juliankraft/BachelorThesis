% Indicate the main file. Must go at the beginning of the file.
% !TEX root = ../main.tex

%%%%%%%%%%%%%%%%%%%%%%%%%%%%%%%%%%%%%%%%%%%%%%%%%%%%%%%%%%%%%%%%%%%%%%%%%%%%%%%%
% 05_conclusion_outlook
%%%%%%%%%%%%%%%%%%%%%%%%%%%%%%%%%%%%%%%%%%%%%%%%%%%%%%%%%%%%%%%%%%%%%%%%%%%%%%%%


\section{Conclusion and Outlook}
\label{conclusion_outlook}

    \subsection{Conclusion}

    \subsection{Outlook}
    - implement other category - is it possible with no data for other or do i need data in order to implement it?

    - generating some more data to make glis\_glis detectable

    - creating a application for the model to make it usable

    - Integrated Gradients - decision analysis


    new data generated info to discuss:
    It is worth noting that in future use cases, the sequence length will likely be more standardized.
    The actual length will depend on the camera settings -- common settings such as 1, 3, 5, or 10 images per trigger -- which can be extracted from the EXIF information of the images.

    Data augmentation a well established way to improve model's generalization \autocite{shortenSurveyImageData2019} was implemented and considered as an option but not actually used in the end.