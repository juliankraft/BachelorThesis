% Indicate the main file. Must go at the beginning of the file.
% !TEX root = ../main.tex

%%%%%%%%%%%%%%%%%%%%%%%%%%%%%%%%%%%%%%%%%%%%%%%%%%%%%%%%%%%%%%%%%%%%%%%%%%%%%%%%
% 04_discussion
%%%%%%%%%%%%%%%%%%%%%%%%%%%%%%%%%%%%%%%%%%%%%%%%%%%%%%%%%%%%%%%%%%%%%%%%%%%%%%%%


\section{Discussion}
\label{discussion}

    \subsection{Detection}
    - some examples for very nice BBoxes and some bad ones

    - discussion of the snail problem

    \subsection{Model Performance}
    - discussing model performance

    - plot with train metrics for all models

    - highlighting the advantages of pretrained models

    \subsection{Best Model}
    - showing the best model and some analysis of its training

    \subsection{Limitations}
    - this brute force approach is dependent on the amount of data available (rare species means less data)


    new data generated info to discuss:
    It is worth noting that in future use cases, the sequence length will likely be more standardized.
    The actual length will depend on the camera settings -- common settings such as 1, 3, 5, or 10 images per trigger -- which can be extracted from the EXIF information of the images.

    Data augmentation a well established way to improve model's generalization \autocite{shortenSurveyImageData2019} was implemented and considered as an option but not actually used in the end.