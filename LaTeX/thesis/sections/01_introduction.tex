% Indicate the main file. Must go at the beginning of the file.
% !TEX root = ../main.tex

%%%%%%%%%%%%%%%%%%%%%%%%%%%%%%%%%%%%%%%%%%%%%%%%%%%%%%%%%%%%%%%%%%%%%%%%%%%%%%%%
% 01-introduction
%%%%%%%%%%%%%%%%%%%%%%%%%%%%%%%%%%%%%%%%%%%%%%%%%%%%%%%%%%%%%%%%%%%%%%%%%%%%%%%%

\section{Introduction}
\label{introduction}

On a global scale loss of biodiversity is one of the most pressing issues of our time \autocite{cardinaleBiodiversityLossIts2012}.
Agricultural expansion, urbanization, and climate change are just a few of the many factors that threaten biodiversity.
For a better understanding of the impact of these factors and the development of effective conservation strategies or simply to track what is being lost it is crucial to monitor biodiversity.
This task has traditionally been done by trained experts, who manually identify species in the field or in images.
Accessible and cheap sensors as well as improved battery and storrage technology have made it possible to deploy large numbers of sensors in the field in order to create vast amounts of images.
The use of Deep Learning (DL) methods to handle has become a standard approach.

    \subsection{Background}
    Biodiversity in Switzerland faces significant pressure due to habitat loss and declining species numbers.
    Particularly, small mammals have received comparatively less attention in conservation research and public perception, despite their crucial ecological roles.
    Of the approximately 30 species of small mammals native to Switzerland, several are threatened and depend on targeted conservation measures.
    Detecting and monitoring small mammals is inherently challenging because these animals are often nocturnal and elusive, dwelling predominantly in dense vegetation or underground burrows.
    Traditional methods for small mammal detection, such as live trapping, are time-consuming, resource-intensive, and pose risks to animal welfare.
    Recent advances in indirect detection methods, including camera traps and environmental DNA (eDNA), offer promising alternatives that are less invasive and potentially more efficient.
    Camera trap systems, specifically optimized for small mammals, generate large volumes of image data, necessitating advanced automated image classification techniques to manage and analyze the data effectively.
    However, distinguishing closely related species visually remains challenging, thereby necessitating complementary genetic identification through collected environmental DNA.
    The integration of eDNA analysis and automated image processing within modular detection systems presents a robust approach to accurately monitor small mammal populations at a broader scale.
    Thus, ongoing research and technological enhancements in both camera trap efficiency and genetic analysis methods are crucial for improving small mammal biodiversity assessments and informing conservation strategies.

