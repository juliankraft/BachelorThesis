% Indicate the main file. Must go at the beginning of the file.
% !TEX root = ../main.tex

%%%%%%%%%%%%%%%%%%%%%%%%%%%%%%%%%%%%%%%%%%%%%%%%%%%%%%%%%%%%%%%%%%%%%%%%%%%%%%%%
% 01_introduction
%%%%%%%%%%%%%%%%%%%%%%%%%%%%%%%%%%%%%%%%%%%%%%%%%%%%%%%%%%%%%%%%%%%%%%%%%%%%%%%%

\section{Introduction}
\label{introduction}

The ongoing loss of biodiversity is among the most urgent environmental issues globally \autocite{brondizioGlobalAssessmentReport2019, cardinaleBiodiversityLossIts2012}.
Small mammals, despite their ecological importance, often receive limited attention in conservation research compared to birds and butterflies \autocite{grafWildlifeCampusKleineSaeugetiere2022}.
This disparity persists although small mammals significantly contribute to ecosystem functions such as seed dispersal and soil aeration.
In Switzerland, of the approximately 30 native small mammal species, several are endangered and require targeted conservation strategies \autocite{bafuListeNationalPrioritaren2019}.

Monitoring small mammal populations traditionally involves labor-intensive methods like live trapping, which not only consume significant time and resources but also pose risks to animal welfare \autocite{grafWildlifeCampusKleineSaeugetiere2022}.
Therefore, innovative methods have emerged, particularly camera trapping combined with environmental DNA (eDNA) sampling, offering non-invasive and efficient alternatives for biodiversity assessment \autocite{aegerterMonitoringKleinmustelidenSchlaefern2019}.

\subsection{Background}

In response to these challenges, the Wildlife@Campus project at the Zurich University of Applied Sciences (ZHAW) focuses on developing advanced methods to systematically detect and monitor small mammals.
The project integrates optimized camera trap systems capable of capturing high-quality images and videos while simultaneously collecting eDNA samples, thereby overcoming the limitations inherent in traditional methods \autocite{grafWildlifeCampusKleineSaeugetiere2022}.
The collected data is subsequently analyzed using automated image classification algorithms, complemented by genetic analyses for precise species identification, particularly important when visual differentiation is insufficient \autocite{ratnaweeraWildlifeCampusProgressReports2021}.

Moreover, the project involves practical measures for habitat enhancement at the ZHAW campus, creating diverse habitats to support various species, thereby actively promoting biodiversity locally.
These interventions are coupled with educational and communication strategies to raise awareness and engagement regarding the ecological roles and conservation needs of small mammals.

\subsection{Problem Statement}

Despite significant advancements, accurately distinguishing visually similar small mammal species using camera traps alone remains challenging.
Additionally, large volumes of image data generated necessitate efficient automated classification tools that can manage data effectively without compromising accuracy.
Thus, the central problem of this thesis is to evaluate and improve deep learning models' effectiveness for small mammal classification from camera trap images, integrating eDNA analyses to validate and complement visual identifications.

\subsection{Related Work}

Deep learning methods, particularly Convolutional Neural Networks (CNNs), have revolutionized image classification tasks, showing significant promise for ecological monitoring.
The MegaDetector \autocite{morrisEfficientPipelineCamera2025}, specifically developed for camera trap images, exemplifies successful application in biodiversity monitoring and provides a foundation for further optimization \autocite{hernandezPytorchWildlifeCollaborativeDeep2024, velezChoosingAppropriatePlatform2022, schneiderRecognitionEuropeanMammals2024}.
Studies such as those by Vélez et al. (2022) and Schneider et al. (2024) demonstrate robust detection performance but highlight limitations in species-level classification, particularly for morphologically similar species.

Integrating eDNA techniques alongside automated image analysis, as explored by recent initiatives, can substantially enhance the accuracy of small mammal detection, providing a more comprehensive and reliable monitoring approach.

This thesis aims to advance these methodologies, contributing to more effective biodiversity conservation strategies.
