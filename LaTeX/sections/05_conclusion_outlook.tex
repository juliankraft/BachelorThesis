% Indicate the main file. Must go at the beginning of the file.
% !TEX root = ../main.tex

%%%%%%%%%%%%%%%%%%%%%%%%%%%%%%%%%%%%%%%%%%%%%%%%%%%%%%%%%%%%%%%%%%%%%%%%%%%%%%%%
% 05_conclusion_outlook
%%%%%%%%%%%%%%%%%%%%%%%%%%%%%%%%%%%%%%%%%%%%%%%%%%%%%%%%%%%%%%%%%%%%%%%%%%%%%%%%


\section{Conclusion and Outlook}
\label{conclusion_outlook}

\subsection{Conclusion}

This thesis demonstrated the effectiveness of deep learning models for detecting and classifying small mammals from camera trap images.
The pretrained EfficientNet-B0 model provided superior classification accuracy, quickly converging and demonstrating robustness across validation folds.
The integration of automated detections utilizing \ac{MD}, while beneficial, revealed some room for improvement, particularly concerning misdetections and missed out detections.
It was found that some finetuning of the \ac{MD} to the specific MammaliaBox camera trap setup could improve the results.
Despite these limitations, the processing pipeline and the trained models provide a promising start for developing an applicable tool and a workflow to reduce manual effort.

Having discussed the main findings, I now return to the initial problem statement outlined at the introduction and address each point with a brief summary.
\begin{itemize}
    \item \textbf{Dataset Preparation:} Review and analyze the existing labeled dataset, including understanding the data structure and labeling conventions.\\
    $\rightarrow$ The existing labeled dataset was analyzed, revealing a structured collection of camera trap images, which were used to train and validate the deep learning models. Labeling conventions were understood and leveraged to ensure correct mapping between images and target classes.
    \item \textbf{Data Processing:} Develop a processing pipeline that includes identification of the \ac{ROI} in each image, selecting appropriate training samples and preparing the data for \ac{DL} model training.\\
    $\rightarrow$ A data processing pipeline was developed to identify \acp{ROI} and prepare training samples. The pipeline incorporated \ac{MD} for automated detection, though some manual preprocessing remained necessary.
    \item \textbf{Model Selection:} Research and select suitable \ac{DL} architectures for the classification task, considering both performance and computational efficiency.\\
    $\rightarrow$ EfficientNet-B0 was selected due to its strong performance and computational efficiency. It outperformed other candidates by achieving high classification accuracy and stable convergence.
    \item \textbf{Model Training:} Train the selected models on the prepared dataset using appropriate optimization strategies.\\
    $\rightarrow$ The selected model was trained on the prepared dataset using standard optimization strategies, including transfer learning and validation across multiple folds, ensuring robust learning.
    \item \textbf{Model Evaluation:} Assess and compare model performance using standard evaluation metrics and conduct a detailed analysis of the most promising architecture.\\
    $\rightarrow$ Model performance was evaluated using standard metrics, and EfficientNet-B0 was found to be the most promising. Its strengths were analyzed in contrast with limitations stemming from detection errors and non-target species.
    \item \textbf{Future pathways:} Discuss the potential for further development and integration of the model into the Wildlife@Campus project, including considerations for future data collection and model refinement.\\
    $\rightarrow$ Future development includes integrating the model into the Wildlife@Campus project, automating sequence extraction via OCR, expanding categories, and exploring sequence-based and temporally aware models to improve accuracy and practical usability.
\end{itemize}

\subsection{Outlook}
Future enhancements should focus on addressing current limitations by introducing an explicit category for non-target species to improve classification accuracy and reduce false predictions with high confidence.
A possible approach could be the \ac{OOD} detection suggested by \textcite{hendrycksBaselineDetectingMisclassified2018}, which could be used to identify images that do not belong to any of the known categories.
To allow for broader application of the model, additional categories would be needed---such as the here explicitly ignored \textit{Glis glis}, a common small mammal in Switzerland on the \textcite{iucnIUCNRedList2025} Red List of Threatened Species.
To improve the model's robustness while adding more categories---possibly with limited data availability---data augmentation techniques could be implemented, as they have been shown to enhance model performance and generalization \autocite{shortenSurveyImageData2019}.

The sequence-based classification did not improve classification performance significantly as it was done in this thesis.
It still seems a very promising approach, since camera trap images are often taken in sequences and the sequence information could be used to improve classification performance.
Further research could explore different options for sequence-based classification.
As a first step, the model's output could be evaluated on the logits level to determine how this information could be utilized to improve classification performance.
More sophisticated approaches could involve utilizing temporally aware models as demonstrated by \textcite{muhammadTemporalSwinFPNNetNovel2024}.
Since the initial detection process using \ac{MD} could still be improved, utilizing sequence information for the detection process could be explored further.
\textcite{zotinANIMALDETECTIONUSING2019} demonstrated a promising non-\ac{DL} approach for detecting animals in camera trap images using sequence information.
There is still a better approach needed to determine the sequence length than the method applied by the Wildlife@Campus team using the \ac{EXIF} timestamp.
Better sequence information is available visually imprinted on top of the images, which could be extracted using an \ac{OCR} model---refer to \autoref{fig:ocr_sample}.
Since the information is imprinted on the images with high contrast and no distortion, this should be a straightforward task for an \ac{OCR} model.
The fact that the first information imprinted on the image is the timestamp---which is also available in the \ac{EXIF} metadata---could be used to match the \ac{OCR} output with the \ac{EXIF} information to quickly determine the reliability of the \ac{OCR} output.

\begin{figure}[ht]
\centering
\includegraphics{figures/ocr_example.pdf}
\caption{Top \(5\%\) of a camera trap image---it was processed with the \texttt{Tesseract} \acs{OCR} model. The output string was: \texttt{2019-09-04 1:02:09 AM M 1/3 \#9 10\textdegree C}.}
\label{fig:ocr_sample}
\end{figure}

Another important step toward practical application is to develop an interface or integrated software solution.
This would allow researchers to actually use the model in their workflows for monitoring small mammal populations.
This could be done in a way that integrates manual review---which is still necessary for reliable results---as a means of further improving the model.
Currently, the data processing pipeline still depends on manual preprocessing of the image metadata to extract sequence information.
This step would benefit from automation to streamline the workflow---ideally, the input for the classification task would be just the raw images as retained from the camera trap.

This thesis builds upon existing efforts by the Wildlife@Campus project and motivates further research in the field of small mammal monitoring using camera traps and automated detection via \ac{DL}.
This exciting field of research has the potential to support wildlife conservation and biodiversity research---in an interdisciplinary and collaborative manner, bringing together ecologists, computer scientists and citizen scientists.
