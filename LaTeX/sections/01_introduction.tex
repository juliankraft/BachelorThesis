% Indicate the main file. Must go at the beginning of the file.
% !TEX root = ../main.tex

%%%%%%%%%%%%%%%%%%%%%%%%%%%%%%%%%%%%%%%%%%%%%%%%%%%%%%%%%%%%%%%%%%%%%%%%%%%%%%%%
% 01_introduction
%%%%%%%%%%%%%%%%%%%%%%%%%%%%%%%%%%%%%%%%%%%%%%%%%%%%%%%%%%%%%%%%%%%%%%%%%%%%%%%%

\section{Introduction}
\label{introduction}

The ongoing loss of biodiversity is among the most urgent environmental issues globally \autocite{brondizioGlobalAssessmentReport2019, cardinaleBiodiversityLossIts2012}.
Small mammals, despite their ecological importance, often receive limited attention in conservation research compared to birds and butterflies \autocite{grafWildlifeCampusKleineSaeugetiere2022}.
This disparity persists although small mammals significantly contribute to ecosystem functions such as seed dispersal and soil aeration.
In Switzerland, of the approximately 30 native small mammal species, several are endangered and require targeted conservation strategies \autocite{bafuListeNationalPrioritaren2019}.
In response to these challenges, the Wildlife@Campus project was initiated to establish a long-term and locally anchored monitoring program for small mammals in the urban context of the \ac{ZHAW} Campus Wädenswil.
By combining citizen science with scientific research, the project aims to improve the detection and understanding of small mammal populations while also raising public awareness and contributing to species conservation efforts \autocite{grafWildlifeCampusKleineSaeugetiere2022}.

\subsection{Background}
Monitoring small mammal populations traditionally involves labor-intensive methods like live trapping, which not only consume significant time and resources but also pose risks to animal welfare \autocite{grafWildlifeCampusKleineSaeugetiere2022}.
Indirect, more efficient survey methods are a much-needed alternative to monitor less invasively and to generate more data with less effort.
A proven effective method is the use of footprint tunnels, which show a very low non-detection rate \autocite{yarnellUsingOccupancyAnalysis2014}.
A further non-invasive method with increasing relevance is the use of \ac{eDNA} for species detection with high reliability \autocite{thomsenEnvironmentalDNAEmerging2015}.
The method in focus for this study utilizes camera traps for species detection --- an approach that has seen rapid and widespread adoption in ecological and conservation research \autocite{delisleNextGenerationCameraTrapping2021}.
Based on findings from the Wildlife@Campus project, a combined approach using camera traps for continuous monitoring and \acs{eDNA}-based methods for species-level identification appears to be a promising and scalable solution for small mammal detection in urban environments \autocite{grafWildlifeCampusKleineSaeugetiere2022}.
As part of the Wildlife@Campus project, initial efforts were made to explore the collection of non-invasive genetic material such as hair and fecal samples.
While these genetic methods were not fully implemented during the pilot phase, they were recognized as essential for identifying cryptic or morphologically similar species and are considered a key component of future monitoring strategies.
Based on the design of the \enquote{Mostela} camera trap box, \textcite{aegerterMonitoringKleinmustelidenSchlaefern2019} developed the \enquote{MammaliaBox}, adapting it to local conditions for monitoring small mustelids and dormice in Switzerland.
In addition to improving image quality, he conducted extensive field testing and produced a detailed guideline for standardized and effective deployment.
During the Wildlife@Campus project, images were generated exclusively using \enquote{MammaliaBoxes} at multiple locations on the ZHAW campus. 
These camera trap boxes produced a substantial volume of image material, capturing a wide range of small mammal activity.
Significant effort has already been invested within the Wildlife@Campus project to manually label a substantial portion of the image dataset and to explore automated approaches to species detection.
In fact, a preliminary solution for training a \ac{DL} model was developed and tested, and a usable labeled dataset already exists \autocite{ratnaweeraWildlifeCampusProgressReports2021}.
However, this work remained largely undocumented and was not brought to full completion. 
In the context of this bachelor's thesis, this dataset will be revisited to explore possibilities to leverage \ac{DL} techniques for species detection in small mammal images.

\subsection{Problem Statement}

The aim of this thesis is to develop a reproducible and effective workflow for the automated classification of small mammals in camera trap images. To achieve this, the following tasks are addressed:

\begin{itemize}
    \item \textbf{Dataset Preparation:} Review and analyze the existing labeled dataset, including understanding the data structure and labeling conventions.
    
    \item \textbf{Data Processing:} Develop a processing pipeline that includes identifying the \ac{ROI} in each image, selecting appropriate training samples and preparing the data for \ac{DL} model training.
    
    \item \textbf{Model Selection:} Research and select suitable \ac{DL} architectures for the classification task, considering both performance and computational efficiency.
    
    \item \textbf{Model Training:} Train the selected models on the prepared dataset using appropriate optimization strategies.
    
    \item \textbf{Model Evaluation:} Assess and compare model performance using standard evaluation metrics and conduct a detailed analysis of the most promising architecture.
\end{itemize}

\subsection{Related Work}

% Deep Learning for image classification
Image classification utilizing \ac{DL} has become a cornerstone of modern computer vision, with significant advancements in recent years.
One of the earliest successful \ac{CNN} architectures was LeNet-5, introduced by \textcite{lecunGradientbasedLearningApplied1998} --- it was designed for handwritten digit recognition and laid the foundation for subsequent developments in \ac{DL} for image classification.
A major breakthrough came with the introduction of AlexNet by \textcite{krizhevskyImageNetClassificationDeep2012}, which demonstrated the power of deep convolutional networks in large-scale image classification tasks.
This architecture won the \ac{ILSVRC} in 2012 and significantly outperformed previous methods, showcasing the potential of \ac{DL} in computer vision.
Following AlexNet, several other influential architectures emerged, including VGGNet \autocite{simonyanVeryDeepConvolutional2015}, GoogLeNet \autocite{szegedyGoingDeeperConvolutions2015} and ResNet \autocite{heDeepResidualLearning2016}.
These architectures introduced various innovations such as deeper networks, inception modules and residual connections, which further improved classification accuracy and model efficiency.
Subsequent models such as DenseNet \autocite{huangDenselyConnectedConvolutional2017} and EfficientNet \autocite{tanEfficientNetRethinkingModel2019} focused on parameter efficiency and scaling strategies, achieving strong performance with fewer computational resources.
The development of these architectures has been complemented by the creation of large-scale datasets like ImageNet \autocite{dengImageNetLargescaleHierarchical2009}, which provided a rich source of labeled images for training and benchmarking \ac{DL} models.
The area of modern \ac{DL} for image classification has continued to evolve with the introduction of transformer-based architectures like \ac{ViT} \autocite{dosovitskiyImageWorth16x162021} and Swin Transformers \autocite{liuSwinTransformerHierarchical2021}, which have shown promising results in various image classification tasks.

% Deep Learning for wildlife monitoring
The success of \ac{DL} in general image classification has led to its increasing application in wildlife monitoring, particularly in the analysis of camera trap images.
The application of \ac{DL} in wildlife monitoring began gaining traction around 2014, when \textcite{chenDeepConvolutionalNeural2014} used \ac{CNN} to classify animal species in camera trap images.
Their approach involved image segmentation via graph-cut algorithms followed by species classification using a \ac{CNN} trained on 14,000 manually segmented images across 20 species.
Building on this early work, \textcite{gomezvillaAutomaticWildAnimal2017} compared various \ac{CNN} architectures on a subset of the Snapshot Serengeti dataset, demonstrating the superiority of deeper models like ResNet-101 for animal identification tasks.
\textcite{norouzzadehAutomaticallyIdentifyingCounting2018} broadened the scope of \ac{DL} applications in wildlife monitoring by developing a model capable of not only identifying but also counting animals in camera trap images.
This work demonstrated the potential of \ac{DL} to automate the labor-intensive task of manual image annotation, significantly improving the efficiency of wildlife monitoring efforts.
Following this, \textcite{tabakMachineLearningClassify2019} applied \ac{DL} techniques to classify species in camera trap images, achieving high accuracy and demonstrating the feasibility of using \ac{DL} for large-scale wildlife monitoring projects.

% use of megadetector
There is a growing number of \ac{AI} platforms and tools designed to facilitate the application of \ac{DL} in wildlife monitoring.
\textcite{velezChoosingAppropriatePlatform2022} provide a comprehensive overview of various platforms including \ac{MD}, \ac{WI}, \ac{MLWIC2} and Conservation \ac{AI}.
\ac{WI} and \ac{MLWIC2} demonstrated low recall—many animals present in images were missed—but high precision for some species.
In contrast, \ac{MD} performed reliably for broad classifications such as distinguishing \enquote{animal} from \enquote{blank}.
The authors conclude that while fully automated species classification is not yet reliable, \ac{DL} tools are valuable for filtering blank images and supporting semi-automated workflows.
\ac{MD}, originally developed by Microsoft, has since become one of the most widely adopted tools for wildlife image filtering and is actively maintained as a standalone detection model \autocite{beeryEfficientPipelineCamera2019}.
Building on this foundation, a highly regarded research team supported by Microsoft is now developing the PyTorch Wildlife project \autocite{hernandezPytorchWildlifeCollaborativeDeep2024}.
This framework incorporates \ac{MD} as a core component and extends its functionality into a full-featured \ac{DL} platform tailored to wildlife monitoring.
A study applying \ac{MD} for image preprocessing is presented by \textcite{schneiderRecognitionEuropeanMammals2024}, who demonstrate its effectiveness in filtering blank images and identifying animals in camera trap datasets.
A particularly notable contribution of their work is the taxonomic classification approach, enabling hierarchical predictions beyond species level to genus, family or order.

% studies focussing on small mammals
While most existing studies have focused on larger mammals, there is a growing interest in applying \ac{DL} techniques to small mammal species.
Camera traps are a proven effective method for monitoring small mammal populations \autocite{clucasCameraTrapMethod2025,aegerterMonitoringKleinmustelidenSchlaefern2019,littlewoodUseNovelCamera2021}.
Despite these developments, there remains a lack of comprehensive \ac{DL} applications specifically targeting small mammal classification.
A notable exception is the work by \textcite{hopkinsDetectingMonitoringRodents2024}, who successfully applied a transfer learning approach to fine-tune a YOLOv5 model for detecting six rodent species from camera trap images.
Their study highlights the feasibility of using \ac{DL} for small mammal monitoring and demonstrates how existing detection models can be adapted to new species with relatively limited training data.
This indicates a promising direction for future research aiming to extend \ac{AI}-powered wildlife monitoring beyond the currently dominant focus on large mammals.

% studies focussing on sequential detection
There are more complex approaches leveraging the often sequential nature of camera trap data to improve detection and classification accuracy.
As an example, \textcite{zotinANIMALDETECTIONUSING2019} utilize sequences for a non-\ac{DL} approach to detect animals in camera trap images taken under complex shooting conditions.
Another study by \textcite{muhammadTemporalSwinFPNNetNovel2024} introduces a sequence-aware and metadata-driven approach to camera trap image classification by adapting the Swin Transformer architecture (Swin-FPN Net) to process image sequences rather than individual frames.
By incorporating temporal dynamics and leveraging metadata such as timestamps, their method enhances classification accuracy and reduces processing time, enabling a more context-sensitive and efficient analysis of wildlife imagery.
