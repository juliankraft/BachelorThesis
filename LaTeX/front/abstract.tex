% Indicate the main file. Must go at the beginning of the file.
% !TEX root = ../main.tex

%%%%%%%%%%%%%%%%%%%%%%%%%%%%%%%%%%%%%%%%%%%%%%%%%%%%%%%%%%%%%%%%%%%%%%%%%%%%%%%%
% Abstract
%%%%%%%%%%%%%%%%%%%%%%%%%%%%%%%%%%%%%%%%%%%%%%%%%%%%%%%%%%%%%%%%%%%%%%%%%%%%%%%%

\vspace*{\fill}
\selectlanguage{english}
\section*{Abstract}
\label{abstract}

This thesis explores the use of \ac{DL} to automate the classification of small mammals captured in camera trap images gathered as part of the Wildlife@Campus project.
A dataset of over 400,000 labeled images, grouped into sequences, was processed using \ac{MD} to filter and crop relevant regions of interest.
Several model architectures were evaluated, with the pretrained EfficientNet-B0 achieving the highest balanced accuracy of \(0.992\) for the classification task.
A comprehensive data pipeline was developed, including detection, preprocessing, cross-validation and classification on an image and sequence level, enabling efficient and reproducible model training and evaluation.
While pretrained models outperformed non-pretrained variants, the results also demonstrated that smaller architectures can be accurate while saving resources.
The study highlights the feasibility of DL for small mammal classification in the given context, the importance of detection quality, label accuracy and the need for a non-target class to handle unknown species such as snails or misdetections such as plant parts or simply empty images.
In addition to the missing non-target class, the study also emphasizes the need for an improved detection process in order to reduce missed sightings.
There is still a high dependency on large amounts of labeled data, which is a real challenge when adding additional classes.
This work lays the foundation for integrating \ac{DL} into the camera trap approach of the Wildlife@Campus project, aiming to reduce the associated manual effort in small mammal monitoring and contribute to ecological research.
\vspace*{\fill}

\newpage

\vspace*{\fill}
\selectlanguage{ngerman}
\section*{Zusammenfassung}

\sloppy{Diese Bachelorarbeit untersucht den Einsatz von \ac{DL} zur automatisierten Klassifikation von Kleinsäugern, die mithilfe von Kamerafallen im Rahmen des Wildlife@Campus-Projekts erfasst wurden.}
Ein Datensatz mit über 400'000 annotierten Bildern, gruppiert in Sequenzen, wurde mithilfe von \ac{MD} verarbeitet, um relevante Bildausschnitte zu filtern und zuzuschneiden.
Mehrere Modellarchitekturen wurden evaluiert, wobei das vortrainierte EfficientNet-B0 die höchste Balanced Accuracy von \(0{,}992\) für die Klassifikationsaufgabe erzielte.
Es wurde eine umfassende Datenpipeline entwickelt, die Erkennung, Vorverarbeitung, Kreuzvalidierung sowie Klassifikation auf Bild- und Sequenzebene umfasst und ein effizientes sowie reproduzierbares Modelltraining und eine entsprechende Evaluation ermöglicht.
Während vortrainierte Modelle gegenüber nicht vortrainierten Varianten bessere Leistungen zeigten, verdeutlichten die Ergebnisse auch, dass kleinere Architekturen präzise sein können und gleichzeitig Ressourcen sparen.
Die Studie betont Anwendbarkeit von \ac{DL} für die Klassifikation von Kleinsäugern in diesem Kontext, die Bedeutung der Detektionsqualität, der Labelgenauigkeit und die Notwendigkeit einer Sonstige-Klasse zur Berücksichtigung unbekannter Arten wie z.B. Schnecken oder Fehldetektionen wie z.B. Pflanzenteile oder leere Bilder.
Neben der fehlenden Sonstige-Klasse unterstreicht die Studie auch den Bedarf nach einem verbesserten Detektionsprozess, um verpasste Sichtungen zu reduzieren.
Zudem besteht weiterhin eine hohe Abhängigkeit von grossen Mengen annotierter Daten, was eine reale Herausforderung bei der Einführung zusätzlicher Klassen darstellt.
Diese Arbeit legt das Fundament für die Integration von \ac{DL} in den Kamerafallen-Ansatz des Wildlife@Campus-Projekts, mit dem Ziel, den damit verbundenen manuellen Aufwand bei der Kleinsäugerüberwachung zu reduzieren und zur ökologischen Forschung beizutragen.
\selectlanguage{english}
\vspace*{\fill}