% Indicate the main file. Must go at the beginning of the file.
% !TEX root = ../main.tex

%%%%%%%%%%%%%%%%%%%%%%%%%%%%%%%%%%%%%%%%%%%%%%%%%%%%%%%%%%%%%%%%%%%%%%%%%%%%%%%%
% Abstract
%%%%%%%%%%%%%%%%%%%%%%%%%%%%%%%%%%%%%%%%%%%%%%%%%%%%%%%%%%%%%%%%%%%%%%%%%%%%%%%%

\vspace*{\fill}

\section*{Abstract}
\label{abstract}

This thesis explores the use of \ac{DL} to automate the classification of small mammals captured in camera trap images as part of the Wildlife@Campus project.
A dataset of over 400,000 labeled images grouped into sequences was processed using \acl{MD} to filter and crop relevant regions of interest.
Several models were evaluated, with the pretrained EfficientNet-B0 achieving the highest balanced accuracy of 0.992 for the classification task.
A comprehensive data pipeline was developed, including detection, preprocessing, cross-validation and classification on an image and sequence level, enabling efficient and reproducible model training and evaluation.
While pretrained models outperformed non-pretrained variants, the results also demonstrated that smaller architectures can be both accurate and resource-efficient.
The study highlights the importance of filtering quality, label accuracy, and the need for a non-target class to handle unknown species such as snails or misdetections such as plant parts or empty images.
Despite high overall performance, limitations remain in detection sensitivity and the need for better handling of underrepresented classes.
This work lays the foundation for integrating \ac{DL} into the efforts of the Wildlife@Campus project, aiming to reduce manual effort in small mammal monitoring and contribute to ecological research.

\vspace*{\fill}
