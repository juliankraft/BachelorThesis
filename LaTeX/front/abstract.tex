% Indicate the main file. Must go at the beginning of the file.
% !TEX root = ../main.tex

%%%%%%%%%%%%%%%%%%%%%%%%%%%%%%%%%%%%%%%%%%%%%%%%%%%%%%%%%%%%%%%%%%%%%%%%%%%%%%%%
% Abstract
%%%%%%%%%%%%%%%%%%%%%%%%%%%%%%%%%%%%%%%%%%%%%%%%%%%%%%%%%%%%%%%%%%%%%%%%%%%%%%%%

\selectlanguage{english}
\section*{Abstract}
\label{abstract}

This thesis explores the use of \ac{DL} to automate the classification of small mammals captured in camera trap images gathered as part of the Wildlife@Campus project.
A dataset of over 400,000 labeled images, grouped into sequences, was processed using \ac{MD} to filter and crop relevant regions of interest.
Several model architectures were evaluated, with the pretrained EfficientNet-B0 achieving the highest balanced accuracy of \(0.992\) for the classification task.
A comprehensive data pipeline was developed, including detection, preprocessing, cross-validation and classification on an image and sequence level, enabling efficient and reproducible model training and evaluation.
While pretrained models outperformed non-pretrained variants, the results also demonstrated that smaller architectures can be both accurate and resource-efficient.
The study highlights the importance of detection quality, label accuracy, and the need for a non-target class to handle unknown species such as snails or misdetections such as plant parts or empty images.
In addition to the missing non-target class, the study also emphasizes the need for an improved detection process in order to reduce missed sightings.
There is still the high dependency on large amounts of labeled data, wich is a real challenge in adding additional classes.
This work lays the foundation for integrating \ac{DL} into the efforts of the Wildlife@Campus project, aiming to reduce manual effort in small mammal monitoring and contribute to ecological research.

\selectlanguage{ngerman}
\section*{Zusammenfassung}
Diese Bachelorarbeit untersucht den Einsatz von \ac{DL} zur automatisierten Klassifikation von Kleinsäugern, die mithilfe von Kamerafallen im Rahmen des Wildlife@Campus Projekts erfasst wurden.
Ein Datensatz mit über 400'000 annotierten Bildern, gruppiert in Sequenzen, wurde mithilfe von \ac{MD} verarbeitet, um relevante Bildausschnitte zu filtern und zuzuschneiden.
Mehrere Modelle wurden evaluiert, wobei das vortrainierte EfficientNet-B0 die höchste Balanced Accuracy von \(0.992\) erzielte.
Es wurde eine umfassende Datenpipeline entwickelt, die Erkennung, Vorverarbeitung, Kreuzvalidierung sowie Klassifikation auf Bild- und Sequenzebene umfasst und ein effizientes und reproduzierbares Modelltraining ermöglicht.
Während vortrainierte Modelle gegenüber nicht vortrainierten Varianten bessere Leistungen zeigten, verdeutlichen die Resultate auch, dass kleinere Architekturen sowohl präzise als auch ressourcenschonend sein können.
Die Studie betont die Bedeutung der Filterqualität, der Labelgenauigkeit sowie die Notwendigkeit einer unbekannten Klasse zur Berücksichtigung unbekannter Arten wie Schnecken oder Fehldetektionen wie Pflanzenteile oder leere Bilder.
Trotz der insgesamt hohen Leistungsfähigkeit bestehen Einschränkungen hinsichtlich der Detektionssensitivität und der Behandlung unterrepräsentierter Klassen.
Diese Arbeit legt das Fundament für den Einsatz von \ac{DL} im Wildlife@Campus Projekt mit dem Ziel, den manuellen Aufwand bei der Kleinsäugerüberwachung zu reduzieren und zur ökologischen Forschung beizutragen.
\selectlanguage{english}