\documentclass{article}
\usepackage[margin=2.5cm]{geometry} % Set margins
\usepackage{graphicx}
\usepackage[absolute]{textpos} % Enable absolute positioning
\usepackage{titlesec} % Package for controlling section title appearance
\usepackage[scaled]{helvet}
\usepackage[T1]{fontenc}
\usepackage{fancyhdr}
\usepackage[utf8]{inputenc}
\usepackage{lmodern}
\usepackage{amsmath}
\usepackage{url} % Load the url package

% Set up colors
\usepackage[dvipsnames]{xcolor} % Color management
% ZHAW Blue: Pantone 2945 U / R0 G100 B166
\definecolor{zhawblue}{rgb}{0.00, 0.39, 0.65}
\definecolor{zhawlightblue}{rgb}{0.82, 0.88, 0.93}

% Set up hyperref
\usepackage{hyperref}
\hypersetup{
    colorlinks=true,
    linkcolor=blue,
    urlcolor=blue,
    citecolor=zhawblue,
    linkbordercolor={0 0 1}
}

% Set up references
\usepackage[
    backend=biber,             % Use biber backend (an external tool)
    sorting=none,              % Enumerates the reference in order of their appearance
    style=apa          % Choose here your preferred citation style
]{biblatex}
\addbibresource{../../biblatex_ba.bib} % The filename of the bibliography

\usepackage[english]{babel} % Set the document language to English

\usepackage[autostyle=true, english=american]{csquotes} 
                               % Required to generate language-dependent quotes 
                               % in the bibliography

\setlength{\TPHorizModule}{1cm} % Set horizontal unit of measure
\setlength{\TPVertModule}{1cm} % Set vertical unit of measure
\setlength{\parindent}{0pt}

\renewcommand{\familydefault}{\sfdefault}

\makeatletter
\renewcommand{\maketitle}{
  \begin{flushleft} 
    \Large\textmd{\@title} 
    \par
  \end{flushleft}
}
\makeatother

% Define style of sectiontitles
\titleformat{\section}
  {\normalfont\large\mdseries}{\thesection}{1em}{}

\titleformat{\subsection}
  {\normalfont\normalsize\itshape} % Adjust style: smaller size, italic
  {} % No label
  {0pt} % Spacing between label and title
  {} % Code to execute after the title
\titlespacing*{\subsection}
  {0pt} % Left margin
  {0.8em} % Space above
  {0.2em} % Space below

% Set up fancyhdr
\fancyhf{} % Clear all headers and footers
\renewcommand{\headrulewidth}{0pt} % Remove the header rule
\rfoot{\thepage} % Place the page number in the right footer
\pagestyle{fancy}

% Add listings package for code highlighting
\usepackage{listings}
\usepackage{xcolor}

%%%%% Title %%%%%
\title{Proposed Workflow for Sequential Based Species Classification of Camera Trap Images}

\begin{document}

%%%%% Header %%%%%
\begin{textblock}{5}(2.5,1) % Position 1cm from left and 1cm from top
  \includegraphics[width=5cm]{logo.jpg} % Add logo
\end{textblock}

\begin{textblock}{6}(13,1) % Position 14cm from left and 1cm from top
        \raggedleft
        Julian Kraft UI22\\
        Proposed Workflow\\
        \today
\end{textblock}

\vspace*{1.5cm}

%%%%% Document %%%%%

\maketitle

%%%%%   %%%%%

\section*{Situation}

The project Wildlife@Campus is a project of the ZHAW and aims to monitor small mustelids in Switzerland.
One approach of the project is using camera traps to collect images of the animals. 
The images are taken using a camera trap box called MammaliaBox especially adapted for this
project \autocite{grafWildlifeCampusKleineSaeugetiere2022}.

The use of cameratraps for wildlifemonitoring is an well established method. Evaluating the large amount of data collected by cameratraps
is providing a resource heavy task -- especially if approached manually. The use of machine learning algorithms 
have proven to be a valuable tool for minimizing the workload. There are several approaches to classify and species and
to detect animals in images. In this project the focus will be to utilize a sequential based approach to classify the species.

\section*{Dataset}

A quite extensive dataset of labeled images is available. The Dataset is divided into 7 sessions. The images are grouped into sequences
of images taken of the same sighting of an animal. The sequences are not of a standardized length and can vary from 

\section*{Data Preparation}



\subsection*{Model Architecture}



\end{document}
