\documentclass{article}
\usepackage[margin=2.5cm]{geometry} % Set margins
\usepackage{graphicx}
\usepackage[absolute]{textpos} % Enable absolute positioning
\usepackage{titlesec} % Package for controlling section title appearance
\usepackage[scaled]{helvet}
\usepackage[T1]{fontenc}
\usepackage{fancyhdr}
\usepackage[utf8]{inputenc}
\usepackage{lmodern}
\usepackage{amsmath}
\usepackage{url} % Load the url package

% Set up colors
\usepackage[dvipsnames]{xcolor} % Color management
% ZHAW Blue: Pantone 2945 U / R0 G100 B166
\definecolor{zhawblue}{rgb}{0.00, 0.39, 0.65}
\definecolor{zhawlightblue}{rgb}{0.82, 0.88, 0.93}

% Set up hyperref
\usepackage{hyperref}
\hypersetup{
    colorlinks=true,
    linkcolor=blue,
    urlcolor=blue,
    citecolor=zhawblue,
    linkbordercolor={0 0 1}
}

% Set up references
\usepackage[
    backend=biber,             % Use biber backend (an external tool)
    sorting=none,              % Enumerates the reference in order of their appearance
    style=apa          % Choose here your preferred citation style
]{biblatex}
\addbibresource{../../biblatex_ba.bib} % The filename of the bibliography

\usepackage[english]{babel} % Set the document language to English

\usepackage[autostyle=true, english=american]{csquotes} 
                               % Required to generate language-dependent quotes 
                               % in the bibliography

\setlength{\TPHorizModule}{1cm} % Set horizontal unit of measure
\setlength{\TPVertModule}{1cm} % Set vertical unit of measure
\setlength{\parindent}{0pt}

\renewcommand{\familydefault}{\sfdefault}

\makeatletter
\renewcommand{\maketitle}{
  \begin{flushleft} 
    \Large\textmd{\@title} 
    \par
  \end{flushleft}
}
\makeatother

% Define style of sectiontitles
\titleformat{\section}
  {\normalfont\large\mdseries}{\thesection}{1em}{}

\titleformat{\subsection}
  {\normalfont\normalsize\itshape} % Adjust style: smaller size, italic
  {} % No label
  {0pt} % Spacing between label and title
  {} % Code to execute after the title
\titlespacing*{\subsection}
  {0pt} % Left margin
  {0.8em} % Space above
  {0.2em} % Space below

% Set up fancyhdr
\fancyhf{} % Clear all headers and footers
\renewcommand{\headrulewidth}{0pt} % Remove the header rule
\rfoot{\thepage} % Place the page number in the right footer
\pagestyle{fancy}

% Add listings package for code highlighting
\usepackage{listings}
\usepackage{xcolor}

%%%%% Title %%%%%
\title{Report on Workflow Research}

\begin{document}

%%%%% Header %%%%%
\begin{textblock}{5}(2.5,1) % Position 1cm from left and 1cm from top
  \includegraphics[width=5cm]{logo.jpg} % Add logo
\end{textblock}

\begin{textblock}{6}(13,1) % Position 14cm from left and 1cm from top
        \raggedleft
        Julian Kraft UI22\\
        Report on Workflow Research\\
        \today
\end{textblock}

\vspace*{1.5cm}

%%%%% Document %%%%%

\maketitle

%%%%%   %%%%%

\section*{Related Research}

The challenge of handling large amounts of images utilizing camera traps for biodiversity seems to be a common problem in the field of ecology.
Approaching the problem by simply working trough the images manually is not feasible due to the sheer amount of data.
Therefore, researchers have developed various methods to automate the process of image classification and species identification.
\textcite{velezChoosingAppropriatePlatform2022} provide an overview of several methods for wildlife image classification, 
including Wildlife Insights (WI), Machine Learning for Wildlife Image Classification (MLWIC2), MegaDetector (MD), and Conservation AI, 
summarizing their respective strengths and limitations.




\section*{Offered Solutions and their Accuracy}


\section*{Challenges}


\section*{Proposed Workflow for this Project}

\begin{enumerate}
  \item In a first step the existing data will be analyzed to get an overview of the available data per label, sequences and their length and the distribution over camera and box types.
  \item All the images will be processed using MD to get the bounding boxes for the animals in the images and exclude images without animals or with more than one bounding box.
  \item This output will be assessed - a protocol to do this is needed.
  \item The data will be split into training and test data keeping complete sequences together.
  \item A small selection of models will be trained on the data -- this selection has to be determined -- to maximize my learning I want to train a model from scratch and I
  want to apply a transfer learning approach.
  \item The models will be evaluated on the test data and the results will be compared.
  \item Depending ond the remaining time budget a separated approach could be explored where whole sequences are classified instead of single images.
\end{enumerate}

\section*{References}

\printbibliography[heading=none]

\end{document}
